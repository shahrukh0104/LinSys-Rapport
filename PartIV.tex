\addcontentsline{toc}{section}{Part IV - State estimation}
\section*{Part IV - State estimation}
\addcontentsline{toc}{subsection}{5.4.1}
\subsection*{5.4.1}
We now have that
\begin{align*}
x &= 
\begin{bmatrix}
    \tilde{p}\\
    \dot{\tilde{p}}\\
    \tilde{e}\\
    \dot{\tilde{e}}\\
    \tilde{\lambda}\\
    \dot{\tilde{\lambda}}
\end{bmatrix},\; u =
\begin{bmatrix}
    \tilde{V_s}\\
    \tilde{V_d}
\end{bmatrix}\; \text{and}\;
y = 
\begin{bmatrix}
    \tilde{p}\\
    \tilde{e}\\
    \tilde{\lambda}
\end{bmatrix}
\intertext{Substituting for the state-space formula we get that}
\dot{x} &= 
\begin{bmatrix}
    \dot{\tilde{p}}\\
    \ddot{\tilde{p}}\\
    \dot{\tilde{e}}\\
    \ddot{\tilde{e}}\\
    \dot{\tilde{\lambda}}\\
    \ddot{\tilde{\lambda}}
\end{bmatrix} = 
A\begin{bmatrix}
    \tilde{p}\\
    \dot{\tilde{p}}\\
    \tilde{e}\\
    \dot{\tilde{e}}\\
    \tilde{\lambda}\\
    \dot{\tilde{\lambda}}
\end{bmatrix} + 
B \begin{bmatrix}
    \tilde{V_s}\\
    \tilde{V_d}
\end{bmatrix}
\\
\begin{bmatrix}
    \dot{\tilde{p}}\\
    \ddot{\tilde{p}}\\
    \dot{\tilde{e}}\\
    \ddot{\tilde{e}}\\
    \dot{\tilde{\lambda}}\\
    \ddot{\tilde{\lambda}}
\end{bmatrix} &= 
\begin{bmatrix}
    0 & 1 & 0 & 0 & 0 & 0\\
    0 & 0 & 0 & 0 & 0 & 0\\
    0 & 0 & 0 & 1 & 0 & 0\\
    0 & 0 & 0 & 0 & 0 & 0\\
    0 & 0 & 0 & 0 & 0 & 1\\
    K_3 & 0 & 0 & 0 & 0 & 0\\
\end{bmatrix}
\begin{bmatrix}
    \tilde{p}\\
    \dot{\tilde{p}}\\
    \tilde{e}\\
    \dot{\tilde{e}}\\
    \tilde{\lambda}\\
    \dot{\tilde{\lambda}}
\end{bmatrix} + 
\begin{bmatrix}
    0 & 0\\
    0 & K_1\\
    0 & 0\\
    K_2 & 0\\
    0 & 0\\
    0 & 0\\
\end{bmatrix}
\begin{bmatrix}
    \tilde{V_s}\\
    \tilde{V_d}\\
\end{bmatrix}
\\
y &= Cx = 
\begin{bmatrix}
    1 & 0 & 0 & 0 & 0 & 0\\
    0 & 0 & 1 & 0 & 0 & 0\\
    0 & 0 & 0 & 0 & 1 & 0
\end{bmatrix}
\begin{bmatrix}
    \tilde{p}\\
    \dot{\tilde{p}}\\
    \tilde{e}\\
    \dot{\tilde{e}}\\
    \tilde{\lambda}\\
    \dot{\tilde{\lambda}}
\end{bmatrix}
\end{align*}



\newpage
\addcontentsline{toc}{subsection}{5.4.2}
\subsection*{5.4.2}
To examine the observability of the system, we need to calculate wether or not the observability matrix has full rank or not. The observability matrix $\mathcal{O}$ is given by

\begin{align*}
\mathcal{O} = \begin{bmatrix}
C\\
CA\\
CA^2\\
CA^3\\
CA^4\\
CA^5
\end{bmatrix}
=
\begin{bmatrix}
1 & 0 & 0 & 0 & 0 & 0\\
0 & 0 & 1 & 0 & 0 & 0\\
0 & 0 & 0 & 0 & 1 & 0\\
0 & 1 & 0 & 0 & 0 & 0\\
0 & 0 & 0 & 1 & 0 & 0\\
0 & 0 & 0 & 0 & 0 & 1\\
0 & 0 & 0 & 0 & 0 & 0\\
0 & 0 & 0 & 0 & 0 & 0\\
K_3 & 0 & 0 & 0 & 0 & 0\\
0 & 0 & 0 & 0 & 0 & 0\\
0 & 0 & 0 & 0 & 0 & 0\\
0 & K_3 & 0 & 0 & 0 & 0\\
0 & 0 & 0 & 0 & 0 & 0\\
0 & 0 & 0 & 0 & 0 & 0\\
0 & 0 & 0 & 0 & 0 & 0\\
0 & 0 & 0 & 0 & 0 & 0\\
0 & 0 & 0 & 0 & 0 & 0\\
0 & 0 & 0 & 0 & 0 & 0
\end{bmatrix}
\end{align*}
As seen $\mathcal{O}$ has full rank, and the system is observable. Next, a linear observer for the system is created
\begin{align*}
\dot{\hat{x}} = A\hat{x} + Bu + L(y - C\hat{x})
\end{align*}
the observer error $e$ is given by
\begin{align*}
e =  x -\hat{x}
\end{align*}
we derive and find that the error estimate satisfy
\begin{align*}
\dot{e}     &= \dot{x} - \dot{\hat{x}}\\
            &= Ax+Bu-A\hat{x}-Bu-L(y-C\hat{x}-Du)\\
            &= Ax - A\hat{x} -Ly + LC\hat{x} + LDu\\
            &= A(x -\hat{x})- L(Cx + Du) + LC\hat{x} + LDu\\
            &= Ae - LC(x - \hat{x})\\
            &= (A - LC)e
\end{align*}
\begin{comment}
When {A,C} is observable an appropriate observer gain L can be chosen by assigning the eigenvalues of A − LC to achieve a certain convergence rate for e.
\end{comment}

The system then equals
\begin{align*}
\begin{bmatrix}
    \dot{\hat{x}}\\
    \dot{e}
\end{bmatrix}\ = \
\begin{bmatrix}
    A-BK     &   BK\\
    0        &   A-LC
\end{bmatrix}
\begin{bmatrix}
    x\\
    e
\end{bmatrix}\ + \
\begin{bmatrix}
    BP\\
    0
\end{bmatrix}r
\end{align*}

\newpage
\addcontentsline{toc}{subsection}{5.4.3}
\subsection*{5.4.3}
If $\tilde{e}$ and $\tilde{\lambda}$ are measured, $C$ is given by
\begin{align*}
C =
\begin{bmatrix}
    0 & 0 & 1 & 0 & 0 & 0\\
    0 & 0 & 0 & 0 & 1 & 0
\end{bmatrix}
\end{align*}
Giving the observability matrix
\begin{align*}
\mathcal{O} =
\begin{bmatrix}
    0 & 0 & 1 & 0 & 0 & 0\\
    0 & 0 & 0 & 0 & 1 & 0\\
    0 & 0 & 0 & 1 & 0 & 0\\
    0 & 0 & 0 & 0 & 0 & 1\\
    K_3 & 0 & 0 & 0 & 0 & 0\\
    0 & K_3 & 0 & 0 & 0 & 0\\
    0 & 0 & 0 & 0 & 0 & 0\\
    0 & 0 & 0 & 0 & 0 & 0\\
    0 & 0 & 0 & 0 & 0 & 0\\
    0 & 0 & 0 & 0 & 0 & 0
\end{bmatrix}
\end{align*}
As there is an element in each column, the matrix has full rank, and the system is observable. If $\tilde{p}$ and $\tilde{e}$ are observed, then
\begin{align*}
C =
\begin{bmatrix}
    1 & 0 & 0 & 0 & 0 & 0\\
    0 & 0 & 0 & 1 & 0 & 0
\end{bmatrix}
\end{align*}
And the observability matrix is given by
\begin{align*}
\mathcal{O} =
\begin{bmatrix}
    1 & 0 & 0 & 0 & 0 & 0\\
    0 & 0 & 1 & 0 & 0 & 0\\
    0 & 1 & 0 & 0 & 0 & 0\\
    0 & 0 & 0 & 1 & 0 & 0\\
    0 & 0 & 0 & 0 & 0 & 0\\
    0 & 0 & 0 & 0 & 0 & 0\\
    0 & 0 & 0 & 0 & 0 & 0\\
    0 & 0 & 0 & 0 & 0 & 0\\
    0 & 0 & 0 & 0 & 0 & 0\\
    0 & 0 & 0 & 0 & 0 & 0
\end{bmatrix}
\end{align*}
This matrix is not of full rank, and the system is not observable.