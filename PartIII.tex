\addcontentsline{toc}{section}{Part III - Multi-variable control}
\section*{Part III - Multi-variable control}
\addcontentsline{toc}{subsection}{5.3.1 Problem 1}
\subsection*{5.3.1 Problem 1}

We have that 
\begin{align*}
    \ddot{\tilde{p}}    &= K_1\tilde{V_d}\\
    \ddot{\tilde{e}}    &= K_2\tilde{V_s}
    \intertext{We want to model in the state-space formulation}
    \dot{x}             &= Ax + Bu
\end{align*}
were
\begin{align*}
x = 
\begin{bmatrix}
    \tilde{p}\\
    \dot{\tilde{p}}\\
    \dot{\tilde{e}}
\end{bmatrix}
\text{and}\ u =
\begin{bmatrix}
    \tilde{V_s}\\
    \tilde{V_d}\\
\end{bmatrix}
\end{align*}
so we substiitute in 
\begin{align*}
\begin{bmatrix}
    \dot{\tilde{p}}\\
    \ddot{\tilde{p}}\\
    \ddot{\tilde{e}}\\
\end{bmatrix} &= 
A\begin{bmatrix}
    \tilde{p}\\
    \dot{\tilde{p}}\\
    \dot{\tilde{e}}
\end{bmatrix}
+ B\begin{bmatrix}
    \tilde{V_s}\\
    \tilde{V_d}
\end{bmatrix}
\end{align*}
and get
\begin{align*}
\begin{bmatrix}
    \dot{\tilde{p}}\\
    \ddot{\tilde{p}}\\
    \ddot{\tilde{e}}\\
\end{bmatrix} &= 
\begin{bmatrix}
    0 & 1 & 0\\
    0 & 0 & 0\\
    0 & 0 & 0
\end{bmatrix}
\begin{bmatrix}
    \tilde{p}\\
    \dot{\tilde{p}}\\
    \dot{\tilde{e}}
\end{bmatrix} +
\begin{bmatrix}
    0 & 0\\
    0 & K_1\\
    K_2 & 0
\end{bmatrix}
\begin{bmatrix}
    \tilde{V_s}\\
    \tilde{V_d}
\end{bmatrix}
\end{align*}


\newpage
\addcontentsline{toc}{subsection}{5.3.2 Problem 2}
\subsection*{5.3.2 Problem 2}
The controllability of the system is examined by calculating the controllability matrix.
\begin{align*}
\mathcal{C} = 
\begin{bmatrix}
\mathbf{B} & \mathbf{AB} & \mathbf{A^2B}
\end{bmatrix} = 
\begin{bmatrix}
0 & 0 & 0 & K_1 & 0 & 0\\
0 & K_1 & 0 & 0 & 0 & 0\\
K_2 & 0 & 0 & 0 & 0 & 0
\end{bmatrix}
\end{align*}
Since the matrix has full rank, the system is controllable. We now wish to apply the multivariable controller
\begin{align*}
\mathbf{u = Pr - Kx}
\end{align*}

Where $\mathbf{Pr}$ is the reference feed-forward and $\mathbf{Kx}$ is the state feedback. The matrix $\mathbf{K}$ corresponds to the linear quadratic regulator (LQR) for which the control input $\mathbf{u =- Kx}$ optimizes the cost function
\begin{align*}
        J = \int_0^\infty\! \mathbf{(x}^T\mathbf{Qx} + \mathbf{u}^T\mathbf{Ru)\, \mathrm{d}x}
\end{align*}

To find the weighting matrices $\mathbf{Q}$ and $\mathbf{R}$ we use Bryson's rule. The different values for $\mathbf{Q}$ were chosen based on the fact that we did not want the pitch ($\tilde{p}$) to be larger than $45^{\circ}$, that is $\frac{\pi}{4}\ rad$, each way; to keep the helicopter stable we did not want the pitch to go from $-45^{\circ}$ to $45^{\circ}$ faster than $0.5$ seconds; and we wished to have a steady climb rate. That gave us 
\begin{align*}
\mathbf{Q} &=
\begin{bmatrix}
    \frac{16}{\pi^2} & 0 & 0\\
    0 & \frac{4}{\pi^2} & 0\\
    0 & 0 & \frac{9}{\pi^2}
\end{bmatrix}
\intertext{In our choice of $\mathbf{R}$, we said that the maximum voltage output should be no more than $10V$, thus}
\mathbf{R} &= 
\begin{bmatrix}
    0.01 & 0\\
    0 & 0.01
\end{bmatrix}
\intertext{Using MATLAB to calculate the matrix $K$, we get}
\mathbf{K} &= 
\begin{bmatrix}
    0 & 0 & 9.55\\
    12.7 & 8.98 & 0
\end{bmatrix}
\intertext{Furthermore, we want to find matrix $\mathbf{P}$ so in theory we get $y = r_c$ as $t\rightarrow \infty$. We have that}
y = 
\begin{bmatrix}
    \tilde{p}\\
    \dot{\tilde{e}}
\end{bmatrix} &= \mathbf{C}x,
\intertext{We Laplace transform the state-space formula, and get}
    s\mathbf{I}x                 &= \mathbf{A}x + \mathbf{B}u\\
\intertext{we now substitue in for u.}
    s\mathbf{I}x                 &= \mathbf{A}x + \mathbf{B}(\frac{\mathbf{P}r_c}{s} - \mathbf{K}x)\\
    s\mathbf{I}x - \mathbf{A}x + \mathbf{B}x       &= \mathbf{B}\frac{\mathbf{P}r_c}{s}\\
    (s\mathbf{I} - \mathbf{A} + \mathbf{BK})x      & =\mathbf{B}\frac{\mathbf{P}r_c}{s}\\
    x                   &= (s\mathbf{I}-\mathbf{A}+\mathbf{BK})^{-1} \mathbf{B}\frac{\mathbf{P}r_c}{s}
\intertext{we substitute} x\ \text{in}\ y &= \mathbf{C}x\\
    y                   &= \mathbf{C}(s\mathbf{I}-\mathbf{A}+\mathbf{BK})^{-1}\mathbf{B}\frac{\mathbf{P}r_c}{s}
\intertext{We use the finit value theorem to find y after $t\rightarrow \infty$ and choose $\mathbf{P}$ such that $y = r_c$. We get}
    y                   &= \lim_{s \to 0} s\mathbf{C}(s\mathbf{I}-\mathbf{A}+ \mathbf{BK})^{-1}\mathbf{B}\frac{\mathbf{P}r_c}{s}\\
    y                   &= \mathbf{C}(-\mathbf{A}+\mathbf{BK})^{-1}\mathbf{BP}r_c
\intertext{For $y$ to be equal to $r_c$, then $\mathbf{P}$ has to equal}
    \mathbf{P}                   &= (\mathbf{C}(-\mathbf{A}+\mathbf{BK})^{-1}\mathbf{B})^{-1}
\intertext{Using MATLAB once more to calculate $\mathbf{P}$, we have that}
    \mathbf{P}          &= \begin{bmatrix}
    0 & 9.55\\
    12.70 & 0
    \end{bmatrix}
\end{align*}
The helicopter improves alot using this system, however we are not able to fly perfectly because of the steady-state error. 


\newpage
\addcontentsline{toc}{subsection}{5.3.3 Problem 3}
\subsection*{5.3.3 Problem 3}
We now modify the previous problem to include an integral effect for the elevation rate and pitch angle, this results in two additional states $\gamma$ and $\zeta$. For which the differential equations are defined as 
\begin{align*}
\dot{\gamma} & = \tilde{p} - \tilde{p_c}\\
\dot{\zeta}  & = \dot{\tilde{e}} - \dot{\tilde{e_c}}
\end{align*}
Hence, the state-vector and input-vector is 
\begin{align*}
x &=
\begin{bmatrix}
    \tilde{p}\\
    \dot{\tilde{p}}\\
    \dot{\tilde{e}}\\
    \gamma\\
    \zeta
\end{bmatrix},\,
\text{and}\
u = 
\begin{bmatrix}
    \tilde{V_s}\\
    \tilde{V_d}
\end{bmatrix}
\intertext{Substituting this in to the state-space formula}
    \dot{x} &= 
\begin{bmatrix}
    \dot{\tilde{p}}\\
    \ddot{\tilde{p}}\\
    \ddot{\tilde{e}}\\
    \dot{\gamma}\\
    \dot{\zeta}
\end{bmatrix} = 
A\begin{bmatrix}
    \tilde{p}\\
    \dot{\tilde{p}}\\
    \dot{\tilde{e}}\\
    \gamma\\
    \zeta
\end{bmatrix} +
B\begin{bmatrix}
    \tilde{V_s}\\
    \tilde{V_d}
\end{bmatrix}\\
\begin{bmatrix}
    \dot{\tilde{p}}\\
    \ddot{\tilde{p}}\\
    \ddot{\tilde{e}}\\
    \dot{\gamma}\\
    \dot{\zeta}
\end{bmatrix} &= 
\begin{bmatrix}
    0 & 1 & 0 & 0 & 0\\
    0 & 0 & 0 & 0 & 0\\
    0 & 0 & 0 & 0 & 0\\
    1 & 0 & 0 & 0 & 0\\
    0 & 0 & 1 & 0 & 0
\end{bmatrix}
\begin{bmatrix}
    \tilde{p}\\
    \dot{\tilde{p}}\\
    \dot{\tilde{e}}\\
    \gamma\\
    \zeta
\end{bmatrix} + 
\begin{bmatrix}
    0 & 0\\
    0 & K_1\\
    K_2 & 0\\
    0 & 0\\
    0 & 0
\end{bmatrix}
\begin{bmatrix}
    \tilde{V_d}\\
    \tilde{V_s}
\end{bmatrix}
\end{align*}
This new matrices $A, B, K$ and $Q$ now equal:
\begin{align*}
A = \begin{bmatrix}
    0 & 1 & 0 & 0 & 0\\
    0 & 0 & 0 & 0 & 0\\
    0 & 0 & 0 & 0 & 0\\
    1 & 0 & 0 & 0 & 0\\
    0 & 0 & 1 & 0 & 0
\end{bmatrix},\qquad
B = \begin{bmatrix}
    0 & 0\\
    0 & K1\\
    K2 & 0\\
    0 & 0\\
    0 & 0
\end{bmatrix},\qquad
C = \begin{bmatrix}
1 & 0 & 0 & 0 &\\
0 & 0 & 1 & 0 &
\end{bmatrix}
\end{align*}
\begin{align*}
K = \begin{bmatrix}
0       & 0     & 16.9 & 0     & 10\\
19.0   & 10.0 & 0     & 10    & 0 
\end{bmatrix}, \qquad
Q &= \begin{bmatrix}
    \frac{16}{\pi^2} & 0 & 0 & 0 & 0\\
    0 & \frac{4}{\pi^2} & 0 & 0 & 0\\
    0 & 0 & \frac{9}{\pi^2} & 0 & 0 \\
    0 & 0 & 0 & 1 & 0 \\
    0 & 0 & 0 & 0 & 1
\end{bmatrix}
\end{align*}

A new state-definition requires a new Q. What we wish to accomplish with the two new states is to reach the reference, without actually effecting the overall stability or transient response. That is why Q is given the values it now has. 

By adding the integral effect the behavior of the helicopter changed significantly. As we expected, the pitch was now able to reach true zero, ie. the steady-state error was dealt with. By introducing the integral effect the elevation rate also became more accurate, and it was easier to control the helicopter as whole. 