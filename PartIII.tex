\addcontentsline{toc}{section}{Part III - Multi-variable control}
\section*{Part III - Multi-variable control}
\addcontentsline{toc}{subsection}{5.3.1}
\subsection*{5.3.1}

We have that 
\begin{align*}
    \ddot{\tilde{p}}    &= K_3\tilde{V_d}\\
    \ddot{\tilde{e}}    &= K_2\tilde{V_s}
    \intertext{We want to model in the state-space formulation}
    \dot{x}             &= Ax + Bu
\end{align*}
were
\begin{align*}
x = 
\begin{bmatrix}
    \tilde{p}\\
    \dot{\tilde{p}}\\
    \dot{\tilde{e}}
\end{bmatrix}
\text{and}\ u =
\begin{bmatrix}
    \tilde{V_s}\\
    \tilde{V_d}\\
\end{bmatrix}
\end{align*}
so we substiitute in 
\begin{align*}
\begin{bmatrix}
    \dot{\tilde{p}}\\
    \ddot{\tilde{p}}\\
    \ddot{\tilde{e}}\\
\end{bmatrix} &= 
A\begin{bmatrix}
    \tilde{p}\\
    \dot{\tilde{p}}\\
    \dot{\tilde{e}}
\end{bmatrix}
+ B\begin{bmatrix}
    \tilde{V_s}\\
    \tilde{V_d}
\end{bmatrix}
\end{align*}


\begin{align*}
\begin{bmatrix}
    \dot{\tilde{p}}\\
    \ddot{\tilde{p}}\\
    \ddot{\tilde{e}}\\
\end{bmatrix} &= 
\begin{bmatrix}
    0 & 1 & 0\\
    0 & 0 & 0\\
    0 & 0 & 0
\end{bmatrix}
\begin{bmatrix}
    \tilde{p}\\
    \dot{\tilde{p}}\\
    \dot{\tilde{e}}
\end{bmatrix} +
\begin{bmatrix}
    0 & 0\\
    0 & K_1\\
    K_2 & 0
\end{bmatrix}
\begin{bmatrix}
    \tilde{V_s}\\
    \tilde{V_d}
\end{bmatrix}
\end{align*}
%%%%%%%%%%%Conclude with something related to the task%%%%%%%%%%%

\newpage
\addcontentsline{toc}{subsection}{5.3.2}
\subsection*{5.3.2}
%%%%%%%%%%Some text with description%%%%%%%%%%%%%%%%%
\begin{align*}
r &= 
\begin{bmatrix}
    \tilde{p_c}\\
    \dot{\tilde{e_c}}
\end{bmatrix}
%%%%%%%%%%%%Something about Brysons rule not being good enough%%%%%%%%%%
\intertext{To find the weighting matrices Q and R we use Brysons rule. We want the pitch ($\tilde{p}$) to be max 45$^{\circ}$, that is $\frac{\pi}{4}$rad. And we want it to accomplish that in $0.5$sek that is $\dot{\tilde{p}} = \frac{\pi}{2}$. In addition to that we want elevation rate to be maximum $\frac{\pi}{6}$. This gives us}
Q &=
\begin{bmatrix}
    \frac{16}{\pi^2} & 0 & 0\\
    0 & \frac{4}{\pi^2} & 0\\
    0 & 0 & \frac{9}{\pi^2}
\end{bmatrix}
\intertext{we say that the maximum voltage output is 10V, so $V_s = V_d = 10V.$ This gives us}
R &= 
\begin{bmatrix}
    0.01 & 0\\
    0 & 0.01
\end{bmatrix}
\intertext{using Matlab, we find that matrix K (The matrix that corresponds to the LQR for which the controll input $u = -Kx$ optimize the cost function) is}
K &= 
\begin{bmatrix}
    0 & 0 & 9.55\\
    12.7 & 8.98 & 0
\end{bmatrix}
\intertext{We want to find matrix P so that in theory we get $y = r_c$ when $t->$(uendelig). We have that}
y = 
\begin{bmatrix}
    \tilde{p}\\
    \dot{\tilde{e}}
\end{bmatrix} 
&= Cx,\,\text{where}\ C = 
\begin{bmatrix}
    1 & 0 & 0\\
    0 & 0 & 0
\end{bmatrix}
\intertext{We Laplace transform the state-space formula, and get}
    sIx                 &= Ax + Bu\\
\intertext{we now substitue in for u.}
    sIx                 &= Ax + B(\frac{Pr_c}{s} - Kx)\\
    sIx - Ax + Bx       &= B\frac{Pr_c}{s}\\
    (sI - A + BK)x      & = B\frac{Pr_c}{s}\\
    x                   &= (sI-A+BK)^{-1} B\frac{Pr_c}{s}
\intertext{we substitute} x\ \text{in}\ y &= Cx\\
    y                   &= C(sI-A+BK)^{-1}B\frac{Pr_c}{s}
\intertext{We use the finit value theorem to find y after t - uendelig and choose P such that $y = r_c$. We get}
    y                   &=\\ %lim s->0% sC(sI-A+ BK)^{-1}B\frac{Pr_c}{s}\\
    y                   &= C(-A+BK)^{-1}BPr_c
\intertext{For $y$ to be equal to $r_c$, then P has to equal}
    P                   &= (C(-A+BK)^{-1}B)^{-1}
\end{align*}
%%%%%%%Something about steady-state error, that will be compusated for in PI%%%%%



\newpage
\addcontentsline{toc}{subsection}{5.3.3}
\subsection*{5.3.3}
We now have:
\begin{align*}
x &=
\begin{bmatrix}
    \tilde{p}\\
    \dot{\tilde{p}}\\
    \dot{\tilde{e}}\\
    \gamma\\
    \zeta
\end{bmatrix},\,
\text{and}\
u = 
\begin{bmatrix}
    \tilde{V_s}\\
    \tilde{V_d}
\end{bmatrix}
\intertext{Substituting this in to the state-space formula}
    \dot{x} &= 
\begin{bmatrix}
    \dot{\tilde{p}}\\
    \ddot{\tilde{p}}\\
    \ddot{\tilde{e}}\\
    \dot{\gamma}\\
    \dot{\zeta}
\end{bmatrix} = 
A\begin{bmatrix}
    \tilde{p}\\
    \dot{\tilde{p}}\\
    \dot{\tilde{e}}\\
    \gamma\\
    \zeta
\end{bmatrix} +
B\begin{bmatrix}
    \tilde{V_s}\\
    \tilde{V_d}
\end{bmatrix}\\
\begin{bmatrix}
    \dot{\tilde{p}}\\
    \ddot{\tilde{p}}\\
    \ddot{\tilde{e}}\\
    \dot{\gamma}\\
    \dot{\zeta}
\end{bmatrix} &= 
\begin{bmatrix}
    0 & 1 & 0 & 0 & 0\\
    0 & 0 & 0 & 0 & 0\\
    0 & 0 & 0 & 0 & 0\\
    1 & 0 & 0 & 0 & 0\\
    0 & 0 & 1 & 0 & 0
\end{bmatrix}
\begin{bmatrix}
    \tilde{p}\\
    \dot{\tilde{p}}\\
    \dot{\tilde{e}}\\
    \gamma\\
    \zeta
\end{bmatrix} + 
\begin{bmatrix}
    0 & 0\\
    0 & K_1\\
    K_2 & 0\\
    0 & 0\\
    0 & 0
\end{bmatrix}
\begin{bmatrix}
    \tilde{V_d}\\
    \tilde{V_s}
\end{bmatrix}
\end{align*}
This gives new matrices $A, B, K$ and $Q$:
\begin{align*}
A = \begin{bmatrix}
    0 & 1 & 0 & 0 & 0\\
    0 & 0 & 0 & 0 & 0\\
    0 & 0 & 0 & 0 & 0\\
    1 & 0 & 0 & 0 & 0\\
    0 & 0 & 1 & 0 & 0
\end{bmatrix},\qquad
B &= \begin{bmatrix}
    0 & 0\\
    0 & K1\\
    K2 & 0\\
    0 & 0\\
    0 & 0
\end{bmatrix},\\
K = \begin{bmatrix}
0       & 0     & 10.52 & 0     & 1\\
19.04   & 10.03 & 0     & 10    & 0 
\end{bmatrix}, \qquad
Q &= \begin{bmatrix}
    \frac{16}{\pi^2} & 0 & 0 & 0 & 0\\
    0 & \frac{4}{\pi^2} & 0 & 0 & 0\\
    0 & 0 & \frac{9}{\pi^2} & 0 & 0 \\
    0 & 0 & 0 & 100 & 0 \\
    0 & 0 & 0 & 0 & 100
\end{bmatrix}
\end{align*}